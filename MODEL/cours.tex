\documentclass{report}
\title{Cours MODEL\thanks{Professeur x, y}}
\usepackage[T1]{fontenc}
%\usepackage[latin1]{inputenc}
\usepackage[utf8]{inputenc} 
\usepackage[french]{babel}
\usepackage{amsfonts}
\usepackage{amsmath}
\newcommand{\Z}{\mathbb{Z}}
\newcommand{\N}{\mathbb{N}}
\newcommand{\Q}{\mathbb{Q}}
\author{Léo Andéol}
\begin{document}
	\maketitle
	\chapter{Bases?}
	\section{Groupes}
	\paragraph{Définition} 
	Soit G un ensemble et $\otimes : G*G \to G$
	On dit que $(G,\otimes)$ est un groupe si :
	\begin{enumerate}
		\item L'opérateur $\otimes$ est associatif $\forall(x,y,z) \in G*G*G$, \newline
		$(x \otimes y)\otimes z = x \otimes (y \otimes z)$
		\item $\forall x \in G, \exists e \in G$ tel que $e \otimes x = x \otimes e = x$,\\
		On appelle cet élément e élément neutre du groupe $(G,\otimes)$
		\item $\forall x \in G, \exists y \in G$ tel que $x \otimes y = e$ \\
		On dira que $y$ est l'inverse de $x$ (on le note parfois $x^-1$)
	\end{enumerate}
	On dira que $G$ est commutatif (ou abélien) si\\
	$\forall(x,y)\in G*G, x\otimes y = y\otimes x$
	\paragraph{Exemples}
	\begin{itemize}
		\item $(\Z,+)$ est un groupe commutatif,\\
		son élément neutre est 0,\\
		$\forall x \in \Z$ l'inverse de $x$ est $-x$
		\item $(\Z,*)$ n'est pas un groupe (car 2 n'a pas d'inverse pour $x$ dans $\Z$)
		\item $(\Q-\{0\},*)$ est un groupe,\\
		1 est l'élément neutre,\\
		$\forall x \in \Q - \{0\}, \frac{1}{x}$ est l'inverse de $x$
		%todo refaire
		\item On considère l'ensemble des bijections $\sum_n$ de $\{1 \dots n\}$ sur $\{1 \dots n\}$\\
		Comme opérateur binaire on choisit la loi de composition $\circ$\\
		Par exemple pour $n=2$ on a :\\
		%todo mettre les deux en largeur
		\[
		\begin{split}
			\phi_0 : 
			\begin{bmatrix}
			1 \to 1 \\ 2 \to 2
			\end{bmatrix}
			\\
			\phi_1 : 
			\begin{bmatrix}
			1 \to 2 \\ 2 \to 2
			\end{bmatrix}
		\end{split}
		\]
		$\phi_1 \circ \phi_0 = \phi_0 \circ \phi_1 = \phi_1$\\
		Donc $(\sum_2,\circ)$ est un groupe
		%todo sigma 3, il faut avancer
		\paragraph{}
		Un groupe $(G, \otimes)$ tel que $G$ est de cardinalité finie est appelé groupe fini.\\
		Pour $p \in \N-\{0\}$,\\
		On note $\frac{\Z}{p\Z}=\{0,1,2,\dots,p-1\}$
		Soit $x \in \Z$ et $(q,r)$ les quotients et le reste de $x$ par $p$ ($0\leq r < p$).\\
		$\frac{\Z}{p\Z}$ contient tous les restes possibles.\\
		Pour $x$ et $y$ dans $\frac{\Z}{p\Z}-\{0\}$ \\
		On note $x \otimes y = x*y \mod p$\\
		$\implies$ on prend le reste de la division euclidienne de $x*y$ par $p$\\
		$\otimes : (\frac{\Z}{p\Z}-\{0\}*\frac{\Z}{p\Z}-\{0\})\to \frac{\Z}{p\Z}-\{0\}$\\
		%todo verifier cours
		Quand $p$ est un nombre premier $\frac{\Z}{3\Z}-\{0\}=\{1,2\}$\\
		$1\otimes 2 = 1*2 \mod 3 = 2 \in \frac{\Z}{3\Z}-\{0\}$\\
		$2\otimes 2 = 4 \mod 3 = 1 \in \frac{\Z}{3\Z}-\{0\}$\\
		$\implies$ On en déduit que $(\frac{\Z}{3\Z}-\{0\},\otimes)$ est un groupe fini/
		\paragraph{}
		Soit $(G,\otimes)$ un groupe fini et $H < G$\\
		On dit que $H$ est stable par $\otimes$ ssi:\\
		$\forall(x,y) \in H*H, x \otimes y \in H$\\
		$\implies$ On peut considérer que la restriction de $\otimes$ à $H$ est un opérateur binaire sur $H$.
		\paragraph{Définition}
		On dit que $(H,\otimes)$ est un sous groupe de $(G,\otimes)$ si
		\begin{enumerate}
			\item $H$ est stable par $\otimes$
			\item $(H,\otimes)$ est un groupe
		\end{enumerate}
		Soit $(G,\otimes)$ un groupe fini\\
		
	\end{itemize}
\end{document}